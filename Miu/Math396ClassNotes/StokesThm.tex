\documentclass[11pt,oneside]{book}



%%%%%%%%%%%%%%Include Packages%%%%%%%%%%%%%%%%%%%%%%%%%%
\usepackage{xcolor}
\usepackage{mathtools}
\usepackage[legalpaper, margin=1in]{geometry}
\usepackage{amsmath}
\usepackage{amssymb}
\usepackage{paralist}
\usepackage{rsfso}
\usepackage{amsthm}
\usepackage{wasysym}
\usepackage[inline]{enumitem}   
\usepackage{hyperref}
\usepackage{tocloft}
\usepackage{wrapfig}
\usepackage{titlesec}
%%%%%%%%%%%%%%%%%%%%%%%%%%%%%%%%%%%%%%%%%%%%%%%%%%%%%%%%


%%%%%%%%%%%%%%%Chapter Setting%%%%%%%%%%%%%%%%%%%%%%%%%%
\definecolor{gray75}{gray}{0.75}
\newcommand{\hsp}{\hspace{20pt}}
\titleformat{\chapter}[hang]{\Huge\bfseries}{\thechapter\hsp\textcolor{gray75}{$\mid$}\hsp}{0pt}{\Huge\bfseries}
%%%%%%%%%%%%%%%%%%%%%%%%%%%%%%%%%%%%%%%%%%%%%%%%%%%%%%%%

%%%%%%%%%%%%%%%%%Theorem environments%%%%%%%%%%%%%%%%%%%
\newtheoremstyle{break}
  {\topsep}{\topsep}%
  {\itshape}{}%
  {\bfseries}{}%
  {\newline}{}%
\theoremstyle{break}
\theoremstyle{break}
\newtheorem{axiom}{Axiom}
\newtheorem{thm}{Theorem}[section]
\renewcommand{\thethm}{\arabic{section}.\arabic{thm}}
\newtheorem{lem}{Lemma}[thm]
\newtheorem{prop}[lem]{Proposition}
\newtheorem{corL}{Corollary}[lem]
\newtheorem{corT}[lem]{Corollary}
\newtheorem{defn}{Definition}[corL]
\newenvironment{indEnv}[1][Proof]
  {\proof[#1]\leftskip=1cm\rightskip=1cm}
  {\endproof}
%%%%%%%%%%%%%%%%%%%%%%%%%%%%%%%%%%%%%%%%%%%%%%%%%%%%%%

%%%%%%%%%%%%%%%%%%%%%%%Integral%%%%%%%%%%%%%%%%%%%%%%%
\def\upint{\mathchoice%
    {\mkern13mu\overline{\vphantom{\intop}\mkern7mu}\mkern-20mu}%
    {\mkern7mu\overline{\vphantom{\intop}\mkern7mu}\mkern-14mu}%
    {\mkern7mu\overline{\vphantom{\intop}\mkern7mu}\mkern-14mu}%
    {\mkern7mu\overline{\vphantom{\intop}\mkern7mu}\mkern-14mu}%
  \int}
\def\lowint{\mkern3mu\underline{\vphantom{\intop}\mkern7mu}\mkern-10mu\int}
%%%%%%%%%%%%%%%%%%%%%%%%%%%%%%%%%%%%%%%%%%%%%%%%%%%%%%



\newcommand{\R}{\mathbb{R}}
\newcommand{\N}{\mathbb{N}}
\newcommand{\Z}{\mathbb{Z}}
\newcommand{\Q}{\mathbb{Q}}
\newcommand{\D}{\mathcal{D}}
\newcommand{\J}{\mathcal{J}}
\newcommand{\T}{\mathcal{T}}
\newcommand{\Td}{\mathcal{T}_d}
\newcommand{\C}{\mathcal{C}}
\newcommand{\M}{\mathcal{M}}
\newcommand{\Lt}{\mathcal{L}}
\newcommand{\Symm}{\text{Symm}}
\newcommand{\A}{\mathcal{A}}
\newcommand{\Alt}{\text{Alt}}
\newcommand{\Complex}{\mathbb{C}}
\newcommand{\Power}{\mathcal{P}}
\newcommand{\ee}{\cdot 10}
\newcommand{\spa}{\text{span}}
\newcommand{\sgn}{\text{sgn}}
\newcommand{\degr}{\text{deg}}
\newcommand{\pd}{\partial}
\newcommand{\that}[1]{\widetilde{#1}}
\newcommand{\lr}[1]{\left(#1\right)}
\newcommand{\vmat}[1]{\begin{vmatrix} #1 \end{vmatrix}}
\newcommand{\bmat}[1]{\begin{bmatrix} #1 \end{bmatrix}}
\newcommand{\pmat}[1]{\begin{pmatrix} #1 \end{pmatrix}}
\newcommand{\rref}{\xrightarrow{\text{row\ reduce}}}


\newcommand{\note}{\color{red}Note: \color{black}}
\newcommand{\remark}{\color{blue}Remark: \color{black}}
\newcommand{\example}{\color{green}Example: \color{black}}
\newcommand{\exercise}{\color{green}Exercise: \color{black}}




%%%%%%%%%%%%table of contents%%%%%%%%%%%%%%%%%%%%%%%%%%%%
\setlength{\cftchapindent}{0em}
\setlength{\cftsecindent}{2em}


\renewcommand\cfttoctitlefont{\hfill\Large\bfseries}
\renewcommand\cftaftertoctitle{\hfill\mbox{}}

\setcounter{tocdepth}{2}
%%%%%%%%%%%%%%%%%%%%%%%%%%%%%%%%%%%%%%%%%%%%%%%%%%%%%%%%%


%%%%%%%%%%%%%%%%%%%%%Footnotes%%%%%%%%%%%%%%%%%%%%%%%%%%%
\newcommand\blfootnote[1]{%
  \begingroup
  \renewcommand\thefootnote{}\footnote{#1}%
  \addtocounter{footnote}{-1}%
  \endgroup
}
%%%%%%%%%%%%%%%%%%%%%%%%%%%%%%%%%%%%%%%%%%%%%%%%%%%%%%%%%

\makeatletter
\def\@seccntformat#1{%
  \expandafter\ifx\csname c@#1\endcsname\c@section\else
  \csname the#1\endcsname\quad
  \fi}
\makeatother


%%%%%%%%%%%%%%%%%%%%%%%%%%%%%%%%%%%Enumerate%%%%%%%%%%%%%%
\makeatletter
% This command ignores the optional argument 
% for itemize and enumerate lists
\newcommand{\inlineitem}[1][]{%
\ifnum\enit@type=\tw@
    {\descriptionlabel{#1}}
  \hspace{\labelsep}%
\else
  \ifnum\enit@type=\z@
       \refstepcounter{\@listctr}\fi
    \quad\@itemlabel\hspace{\labelsep}%
\fi}
\makeatother
\parindent=0pt
%%%%%%%%%%%%%%%%%%%%%%%%%%%%%%%%%%%%%%%%%%%%%%%%%%%%%%%%%%
\begin{document}


\begin{thm}[The Generalized Stokes' Theorem]
Let $k>1$, let $M$ be a compact oriented $k$-manifold in $\R^n$, with $\partial M$ having the induced orientation if $\partial M$ is not empty, let $\omega$ be a $(k-1)$-form defined in an open set of $\R^n$ containing $M$, then we have the following holds if $\partial M$ is not empty:
\begin{align*}
\int_M d\omega  = \int_{\partial M}\omega
\end{align*}
and we have the following holds if $\partial M$ is empty:
\begin{align*}
\int_{\partial M}\omega = 0
\end{align*}
\end{thm}
\begin{proof}
A detailed proof is provided on Munkres Theorem 37.2. Here it suffices to prove the special case where $\text{supp}(\omega) \subseteq V$ with $\alpha:U \to V$ being a coordinate patch on $M$. Note that $\text{supp}(d\omega) \subseteq \text{supp}(\omega)$, hence we have $\text{supp}(\omega) \subseteq V$. The general case will follow by taking finite sums over regions on $M$ parametrized by coordinate patches. \\

First we note that we can extend the definition of domain of $d(\alpha^*\omega)$ to $\mathbb{H}^k$
\begin{align*}
\int_M d\omega = \int_U \alpha^* d\omega = \int_U d(\alpha^*\omega) = \int_{\mathbb{H}^k} d(\that{\alpha^*\omega})
\end{align*} 
On the other hand, we can write:
\begin{align*}
\int_{\pd M}\omega = \int_{U \cap \pd \mathbb{H}^k}\alpha^*\omega  = \int_{\pd \mathbb{H}^k} \that{\alpha^*\omega}
\end{align*}
Write $d\that{\alpha^*\omega} = f_1 \, dx_2 \wedge dx_3\wedge \cdots \wedge dx_k+ f_2 dx_1 \wedge dx_3 \wedge \cdots \wedge dx_k + \cdots + f_k \wedge dx_1 \wedge dx_2 \wedge \cdots \wedge dx_{k-1}$. Here we have:
\begin{align*}
d\that{\alpha^*\omega} = (D_1f_1 - D_2f_2 + \cdots + (-1)^{k-1}D_kf_k)dx_1 \wedge dx_2 \wedge \cdots \wedge dx_k
\end{align*}
\begin{align*}
\int_{\mathbb{H}^k} d\that{\alpha^*\omega} &= \int_{\mathbb{H}^k}(D_1f_1 - D_2f_2 + \cdots + (-1)^{k-1}D_kf_k) \\
&= \int_B (D_1f_1 - D_2f_2 + \cdots + (-1)^{k-1}D_kf_k)
\end{align*}
where $B$ is a box $a_1 \leq x_1 \leq b_1 , \cdots, 0 = a_k \leq x_k \leq b_k$. Here by Fubini's Theorem and Fundamental Theorem of Calculus, we have:
\begin{align*}
\int_{\mathbb{H}^k} d\that{\alpha^*\omega} &= \int_B (D_1f_1 - D_2f_2 + \cdots + (-1)^{k-1}D_kf_k)\\
&= 0 - 0 +0 - 0 +\cdots + (-1)^{k-1} \int_{\mathbb{H}^k} D_k f_k\\
&= (-1)^{k-1}\int_{\R^{k-1} } f_k(x_1, x_2,\cdots, x_{k-1}, 0)\\
&= \int_{\pd \mathbb{H}^k} f_k dx_1 \wedge \wedge dx_2  \wedge \cdots \wedge dx_{k-1} \\
&= \int_{\pd \mathbb{H}^k} \alpha^*\omega
\end{align*}
The result follows.
\end{proof}

\end{document}